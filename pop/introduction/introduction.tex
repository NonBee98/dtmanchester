\chapter{Introduction}

\section{Overview}
\label{sec:overview}
As one of the most popular financial products, stock takes high risks for high rewards. Its price is affected by innumerous uncertain factors such as policies of local government, breaking of diseases, etc. \cite{zhang2017stock}. To find strategies that can maximize profits and reduce risks, stock investors may need to extract the latent trading and price moving patterns of stock market \cite{hu2015stock}.
However, due to the randomness, complexity and uncertainty existing in stock market, those patterns are elusive and hard to be discovered \cite{li2018stock}.  \cite{basu2010use} states that even though researchers made massive efforts to teaching machine to recognize data patterns automatically in the past decades, humans are still the best pattern recognizers in most cases so far. Based on whether expert derived patterns exists (and being used) or not, pattern discovery methods can be classified into two categories: supervised methods and unsupervised methods \cite{fu2001pattern}. In terms of stock price, supervised methods require fluctuation patterns pre-defined by experts, while unsupervised methods do not need prior knowledge of interesting structures. Consider that stock market is dynamic and rapidly developing and stock price series shows no periodic trajectory (at least in the short term) \cite{wei2012exotic}, the pre-defined patterns may need to be updated frequently (relies heavily on experts). Increasing stock price data volume makes manually defining patterns much more challenging \cite{basu2010use}. These problems make stock researchers such as \cite{zhang2011novel} focus more on unsupervised pattern discovery methods.

\section{Motivation}
The majority of existing stock-related researches aim to predict the stock price or moving trending. However, \cite{wei2012exotic} already shown that stock pirce is unpredictable at least in the short term. Instead of directly predicting the stock pirce, we hope to retrieve the latent stock price moving patterns to help decision making. As mentioned in section \ref{sec:overview} supervised methods require domain knowledge, therefore, we seek help from unsupervised methods. Most previous works such as \cite{fu2001pattern,zhang2011novel} mainly try to modify clustering algorithms to get better performance. They use the original data (or part of the data) as the input vector, ignore the usage of representation learning. In this project, we will adapt some novel time-series data representation methods, especially the path signature feature (PSF) used in \cite{lyons2014rough,boedihardjo2016signature,chen1958integration}, combined with traditional clustering algorithms to better reveal the hidden fluctuation patterns of stock pirce.

\section{Aims and Objectives}

The general goal of this project is to categorize different stock price trajectories into several classes. It can be detailed as follows:
\begin{enumerate}
    \item Finding a proper representation method that can reduce the data dimension and better reveal the feature of a stock price trajectory.
    \item Grouping those generated feature vectors to get the final set of hidden patterns based on clustering algorithms.
\end{enumerate}
To achieve the goal, following works are need to be done:
\begin{enumerate}
    \item Review the recent works related to time-series data representation and clustering algorithms.
    \item Collect stock price data from open-source websites. 
    \item Preprocess data. In this process, we will apply the common numerical data processing pipeline to our data set.
    \item Apply data representation algorithms to our data, analyze the effectiveness of those transformation in different aspects. This may require numerical analysis and visualization.
    \item Apply multiple clustering algorithms to the generated representation.
    \item Evaluate and compare the performance of those clustering algorithms based on certain metrics.
    \item Visualize the patterns generated by clustering algorithms.
    \item Integrate pattern matching algorithm into this project. Apply it to test set for the final evaluation.
\end{enumerate}


\section{Description of the Work}
\textbf{Project scope:}\\
In this project, we will analyze some of the US-based stocks trading on the NYSE, NASDAQ, and NYSE. In our dataset, the ealierest records could date back to 1970s, and it's hard to analyze the whole series. Instead, we will split their historical series into smaller fragments. To get more stable patterns, we mainly focus on the long-term price fluctuation, meaning that each smaller fragment will cover more than 6 months data. Data compression and representation algorithms will be applied to each fragment to generate the feature vectors of them. Then those feature vectors will be grouped by clustering algorithms, the centroid of each group is regarded as a unique pattern. Finally, the value of the generated patterns will be examined by using pattern matching methods to checking whether similar patterns can be found in test data.
\\
\textbf{The final deliverables of this project will be:}
\begin{enumerate}
    \item The data set used in this project
    \item The code implementation of this project
    \item The dissertation with experimental results, analysis and visualization of this project
\end{enumerate}

\section{Report Structure}
The remaining chapters are organised as follows: Chapter \ref{ch:background} briefly introduces works related to time-series data representation and clustering. Chapter \ref{ch:Methodology} shows the detailed methodology used in this project. Chapter \ref{ch:ethics} includes the ethical consideration of this project. Chapter \ref{ch:risk} evaluates the potential risks of this project. Chapter \ref{ch:evaluation} introduces some metrics that can be used to evaluate the performance of the algorithms used in this project. Chapter \ref{ch:plan} includes our progress and future work.
